% Chapeau (Introduction) - Version Francaise
% Il s'agit de l'ouverture de la these - un bref apercu avant d'entrer dans les details

% =============================================================================
% STRUCTURE DU CHAPEAU
% =============================================================================
% Le chapeau se compose generalement de 3 a 5 paragraphes qui:
% 1. Posent le contexte general (domaine, champ d'etude)
% 2. Resserrent vers le probleme specifique
% 3. Enoncent ce que fait cette these
% 4. Presentent l'approche/les contributions
% 5. Guident le lecteur vers la suite
% =============================================================================

% --- PARAGRAPHE 1: Contexte General ---
% Quel est le domaine general?
% Pourquoi est-ce important? (pertinence societale, scientifique, industrielle)
%
% [A REDIGER]

% --- PARAGRAPHE 2: Resserrement vers le Probleme ---
% Quel defi/manque specifique existe?
% Quelles sont les limites des approches actuelles?
%
% [A REDIGER]

% --- PARAGRAPHE 3: Cette These ---
% Que propose/etudie ce travail?
% Quelle est l'idee ou l'approche centrale?
%
% [A REDIGER]

% --- PARAGRAPHE 4: Apercu des Contributions ---
% Quelles sont les contributions principales? (bref, haut niveau)
% Qu'apprendra le lecteur?
%
% [A REDIGER]

% --- PARAGRAPHE 5: Plan du Document (optionnel) ---
% Comment ce document est-il organise?
% Que couvrira chaque chapitre?
%
% [A REDIGER]
