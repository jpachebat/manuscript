\newcommand{\newjp}[1]{\textcolor{cyan}{#1}}

\newcommand \comment[1]{{\color{orange}[#1]}}
\newcommand \new[1]{{\color{blue}#1}}
\newcommand \review[1]{{\color{orange}#1}}

\newcommand{\todojp}[1]{\todo[backgroundcolor=yellow!20!white,inline]{#1}}
\newcommand{\red}[1]{{\color{red} #1}}
\newcommand \bleu[1]{{\color{blue}#1}} 
\definecolor{darkblue}{rgb}{0,0,.4}
\newcommand \ok[1]{{\color{darkblue}#1}} % le texte qui est validé
\newcommand \nch[1]{{\color{orange}#1}} % pour reperer les chgts d'Emmanuel

% Algorithmic compatibility
\providecommand{\State}{\STATE}

\renewcommand{\P}{{\mathbb{P}}}

\newcommand{\ve}{\varepsilon}
\newcommand{\eqdist}{\overset{d}{=}}
\newcommand{\cB}{\mathcal{B}}
\newcommand{\calR}{\mathcal{R}} % or use \mathcal{R} in the body

% Abbreviations
\newcommand\stdf{stdf\;}

\newcommand\df{$df$}
\newcommand\lit{\textit{l}\;}

%\newcommand{\N}[1]{\mathbb{N}^{#1}}
\newcommand{\E}{\mathbb{E}}
\renewcommand{\P}{\mathbb{P}}
%\newcommand{\R}[1]{\mathbb{R}^{#1}}
\newcommand \du{\dd u}
\newcommand \dv{\dd v}
\newcommand \dx{\dd x}
\newcommand \dy{\dd y}

\newcommand{\KL}{\mathrm{KL}}

\newcommand{\mathbi}[1]{\textbf{\textit{#1}} }
\newcommand{\an}{\mathbi{a}_n}
\newcommand{\bn}{\mathbi{b}_n}
\renewcommand{\at}{\mathbi{a}_t}  % Override babel's \at
\newcommand{\abf}{\mathbi{a}}
\newcommand{\bbf}{\mathbi{b}}
\newcommand{\ei}{\mathbi{e}_i}
\newcommand{\mbf}{\mathbi{m}}
\newcommand{\Mbf}{\mathbi{M}}
\newcommand{\Mn}{\mathbi{M}_n}
\newcommand{\ubf}{\mathbi{u}}
\newcommand{\xbf}{\mathbi{x}}
\newcommand{\Zbf}{\mathbi{Z}}
\newcommand{\Gbf}{{\boldsymbol{\Gamma}}}
\newcommand{\alphabf}{\boldsymbol{\alpha}}
\newcommand{\Ubf}{\mathbi{U}}
\newcommand{\Xbf}{\mathbi{X}}
\newcommand{\Ybf}{\mathbi{Y}}
\newcommand{\Ibf}{\mathbi{I}}

\newcommand{\ombf}{\boldsymbol{\omega}}
\newcommand{\gammabf}{\boldsymbol{\gamma}}
\newcommand{\zeros}{\mathbf{0}}
\newcommand{\ones}{\mathbf{1}}
\newcommand{\albf}{\boldsymbol{\alpha}}
\newcommand{\nubf}{\boldsymbol{\nu}}
\newcommand{\thetabf}{\boldsymbol{\theta}}
\newcommand{\Hbb}{\mathbb{H}}
\newcommand{\Ccal}{\mathcal{C}}
\newcommand{\Dcal}{\mathcal{D}}
\newcommand{\Gcal}{\mathcal{G}}
\newcommand{\Xcal}{\mathcal{X}}
\newcommand{\Ycal}{\mathcal{Y}}
\newcommand{\Thetabf}{\boldsymbol{\Theta}}
\newcommand{\alphabb}{\boldsymbol{\alpha}}



%\newcommand{\mda}[1]{\mathcal{D}\left(#1\right)}
\newcommand{\mda}[1]{\mathrm{Dom}\left(#1\right)}
\newcommand{\tendsto}[1]{\underset{#1}{\longrightarrow}}
\newcommand{\asc}{\overset{\mbox{\em a.s.}}{\longrightarrow}}
\newcommand \ndata{{n_{\rm data}}}
\newcommand \nsample{{n_{\rm sample}}}
\newcommand \pdata{{p_{\rm data}}}
\newcommand \ake{\sc ake}

% Figure caption helpers (customizable by \renewcommand)
\newcommand{\twodOverlayCaption}{Overlay of real and generated (mean--scale). \texttt{n\_samples=1,000}.}

%%%%%%%%%%%%%%%%%%%%%%%%%%%%%%%%%%%%%%%%%%%%%%%%%%%%%%%%%%%%%%%%%%%%%%%%%%%%%%%
% Diffusion model notation (aligned with ICLR 2026 paper)
%%%%%%%%%%%%%%%%%%%%%%%%%%%%%%%%%%%%%%%%%%%%%%%%%%%%%%%%%%%%%%%%%%%%%%%%%%%%%%%

% Schedule functions
\newcommand{\alphat}{\alpha_t}
\newcommand{\sigmat}{\sigma_t}
\newcommand{\alphas}{\alpha_s}
\newcommand{\sigmas}{\sigma_s}

% Denoiser and noise predictor
\newcommand{\denoiser}[2]{\hat{x}_0^{#1}(#2)}  % \denoiser{\theta}{x_t, t}
\newcommand{\noisepred}[2]{\hat{x}_1^{#1}(#2)} % \noisepred{\theta}{x_t, t}

% Score function
\newcommand{\score}[2]{s^{#1}(#2)}  % \score{\theta}{x_t, t}

% Forward kernel
\newcommand{\fwdkernel}{q_{t|0}}

% DDIM variance schedule
\newcommand{\etat}{\eta_t}
\newcommand{\etas}{\eta_s}

% Base and tilted distributions
\newcommand{\pbase}{p_0^{\mathrm{b}}}
\newcommand{\ptilt}[1]{p_0^{#1}}  % \ptilt{\beta r} for tilted distribution
\newcommand{\pmargin}[1]{p_{#1}}  % \pmargin{t} for marginal at time t

% Iterative tilting notation
\newcommand{\piter}[1]{p_0^{#1:N}}  % \piter{k} for k-th iterative tilt
\newcommand{\pitermargin}[2]{p_{#1}^{#2:N}}  % \pitermargin{t}{k} for marginal

% Normalising constant
\newcommand{\normconst}[1]{Z_{#1}}

% Covariance operator
\newcommand{\Cov}{\mathrm{Cov}}

% IID sampling notation
\newcommand{\simiid}{\overset{\mathrm{iid}}{\sim}}

% Additional macros from HTGAN/GENOS papers (using providecommand to avoid conflicts)
\providecommand\rvs{r.v.s\;}
\providecommand\Prob[1]{\P\left(#1\right)}
\providecommand\Esp[1]{\E\left[#1\right]}
\providecommand\NN{\mbox{NN}}
\providecommand\sumi{\sum_{i\in[d]}}
\providecommand{\Thbb}{\boldsymbol{\Theta}}
\providecommand\maxi{\max_{i\in[d]}}
\providecommand\mini[1]{\min\left\{#1\right\}}
\providecommand\supp{\mbox{supp}}

% JMLR compatibility
\providecommand{\acks}[1]{\paragraph{Acknowledgements.} #1}

% Float barrier
\usepackage{placeins}
